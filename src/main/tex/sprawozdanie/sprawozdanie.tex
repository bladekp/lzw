% sprawozdanie.tex

% Piotr Błądek

\documentclass{article}
\usepackage{times}
\usepackage{polski}
\usepackage[utf8]{inputenc}
\usepackage{hyperref}
\usepackage{listings}

\begin{document}

% Article top matter
\title{Sprawozdanie - implementacja algorytmu LZW} 
\author{
        Wojciech Decker \texttt{deckerw@ee.pw.edu.pl} \\
        Piotr Błądek \texttt{bladekp@ee.pw.edu.pl} \\
        Politechnika Warszawska \\
        Wydział Elektryczny \\
        Kompresja Danych Projekt
       }
\date{\today} 
\maketitle

\section{Wstęp}

Projekt zakładał napisanie implementacji algorytmu LZW do bezstratnej kompresji danych
i przetestowanie go dla różnych danych, z porównaniem sprawnośi, czyli tego ile
miejsca udało się przy pomocy LZW zaoszczędzić.

\section{Struktura}

Projekt napisany w języku JAVA, budowany przy pomocy narzędzia maven. Podział projektu:
\begin{itemize}
    \item \texttt{src/main/java} - kody źródłowe JAVA
    \item \texttt{src/main/resources} - zasoby, w szczególności dane do testowania
    \item \texttt{src/main/tex} - dokumentacja, w szczególności sprawozdanie
\end{itemize}
Kody źródłowe składają się z trzech klas. Klasa Application zawiera metodę main,
służy do interakcji poprzez CLI z użytkownikiem. Pozostałe dwie klasy to LZWCompress,
oraz LZWDecompress, ich nazwy mówią same do czego te klasy służą. Repozytorium 
projektu znajduje się w lokalizacji: \url{https://github.com/bladekp/lzw/tree/master}.

\section{Budowanie}

Projekt, jak już wcześniej zaznaczyłem, budujemy przy pomocy narzędzia maven. Po
ściągnięciu kodów źródłowych wykonujemy polecenie:
\begin{lstlisting}
mvn clean install 2>error.log
\end{lstlisting}
Po wykonaniu tego polecenia pojawi się katalog target, w którym to znajdziemy między innymi niniejszy 
dokument oraz archiwum LZW.jar które zawiera skompilowany, gotowy do uruchomienia program.

\section{Uruchamianie}

Program jest uruchamiany z konsoli w następujący sposób:
\begin{lstlisting}
java -jar lzw.jar OPCJA [ZRODLO] [CEL]
\end{lstlisting}
Gdzie ZRODLO to plik wejściowy, CEL plik wyjściowy, a OPCJA to -k w przypadku 
kompresji pliku żródłowego do decelowego lub -d w przypadku dekompresji pliku 
żródłowego do decelowego. Wszystkie parametry są wymagane. Można też użyć 
programu z opcją -h w celu pokazania pomocy. 

\section{Testy}

Przeprowadzonych zostało kilka testów, w poniższej tabeli przedstawiam ich wyniki:

\begin{center}
	\begin{tabular}{| l | l | l | l | l | l |} 
		\hline  
		Typ     & Rozmiar       & Rozmiar       & Czas          & Czas              & Stosunek wielkości    \\ 
                pliku   & wejściowy [B] & wyjściowy [B] & kompresji [s] & dekompresji [s]   & (po/przed) [\%]       \\
                \hline
		tekstowy    & 2298      & 1598  & 0,081 & 0,099 & 69,5  \\
                tekstowy    & 73882     & 26025 & 0,198 & 0,140 & 35,2  \\
                bitmapa     & 1738      & 1046  & 0,111 & 0,092 & 60,2  \\
                bitmapa     & 1043403   & 18969 & 0,913 & 0,114 & 1,8   \\
		\hline
	\end{tabular}
\end{center}

Czas był mierzony przy pomocy linuxowej komendy time. Pliki miały różną zawartość
pierwszy plik tekstowy to 3 akapity tekstu Lorem Ipsum, drugi, większy plik,
to manual polecenia find. Pierwszy plik graficzny to prosty obraz 20x20 pikseli,
dwa kolory. Drugi obraz jest większy 640x480 pikseli, zawiera sporo informacji.

\section{Podsumowanie}

Jak można zauważyć czasy kompresji nie są duże, zaoszczędzonego miejsca jest sporo,
szczególnie widać to w przypadku większego pliku graficznego gdzie oszczędziliśmy ponad 
98\% miejsca na dysku (dla przypomnienia LZW to kompresja bezstratna, czyli po 
odkompresowaniu nie stracimy żadnej informcji). LZW w podstawowej postaci to prosty
w implementacji algorytm. Do poprawnego produkcyjnego działania nasza implementacja
wymagałaby jednak jeszcze kilku poprawek.

\end{document} 
